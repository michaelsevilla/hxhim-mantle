\section{ParSplice Background}
\label{sec:parsplice}

ParSplice~\cite{perez:jctc20150parsplice} is an accelerated molecular dynamics
(MD) simulation package developed at LANL. It is part of the Exascale Computing
Project\footnote{http://www.exascale.org/bdec/} and is important to LANL's
Materials for the Future initiative. Its phases are:

\begin{enumerate}

  \item a splicer tells workers to generate segments (short MD trajectory) for
  specific states

  \item workers read initial coordinates for their assigned segment from data
  store; the key-value pair is (state ID, coordinate)

  \item upon completion, workers insert final coordinates for each segment into
  data store, and wait for new segment assignment

\end{enumerate}

The computation can be parallelized by adding more workers or by adding worker
tasks to parallelize individual workers.  The workers are stateless and read
initial coordinates from the data store each time they begin generating
segments. Since worker tasks do not maintain their own history, they can end up
reading the same coordinates repeatedly. To mitigate the consequences of these
repeated reads, ParSplice provisions a hierarchy of nodes to act as caches that
sit in front of a single node persistent database.  Values are written to each
tier and reads traverse up the hierarchy until they find the data. 

We use ParSplice to simulate the evolution of metallic nanoparticles that grow
from the vapor phase. As the run progresses, the energy landscape of the system
becomes more complex. Two domain factors control the number of entries in the
data store: the growth rate and the temperature. The growth rate controls how
quickly new atoms are added to the nanoparticle: fast growth rates lead to
non-equilibrium conditions, and hence increase the number of states that can be
visited. However, as the particle grows, the simulation slows down because the
calculations become more expensive, limiting the rate at which new states are
visited. On the other hand, the temperature controls how easily a trajectory
can jump from state to state; higher temperatures lead to more frequent
transitions.

The nanoparticle simulation stresses the data store architecture of ParSplice.
It visits more states than other input decks because the system uses a cheap
potential, has a small number of atoms, and operates in a complex energy
landscape with many accessible states. Changing growth rates and temperature
alters the size, shape, and locality of the data store keyspace. Lower
temperatures and smaller growth rates create hotter keys with smaller keyspaces
as many segments are generated in the same set of states before the trajectory
can escape to a new region of state space.

Our evaluation uses the total ``trajectory length" as the goodness metric. This
value is the duration of the overall trajectory produced by ParSplice. At
ideal efficiency, the trajectory length should increase with the square root of
the wall-clock time, since the wall-clock cost of time-stepping the system by
one simulation time unit increases in proportion of the total number of atoms.
