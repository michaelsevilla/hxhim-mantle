\section{Conclusion}

Load balancing is a well-know technique for improving storage system
performance, yet finding the best policies is a difficult, multi-dimensional
problem. Rather than attempting to construct a single, complex load balancing
policy that works for a variety of scenarios, we instead use the Mantle
framework as a \emph{microservice} approach to load balancing that enables
software-defined storage systems to flexibly change policies as the workload
changes over time.  In our analysis of the ParSplice key-value workload we have
detected clear workload regimes that are sensitive to the initial conditions
and the scale and duration of the simulation. We have also demonstrated that
changing load balancing policies at runtime in response to the current workload
is an effective mechanism to providing better load distribution.  Finally, we
have demonstrated that the classification strengths of many machine learning
algorithms is an effective mechanism for detecting access pattern regimes
within the ParSplice application's key-value store usage.

%Our next steps are to integrate Mantle and the HXHIM distributed key-value
%storage service into ParSplice and dynamically select from a variety of load
%balancing policies based on the current access regime.  More speculatively, we
%hope to leverage machine learning models that assist in the selection or even
%creation of those load balancing policies.  
